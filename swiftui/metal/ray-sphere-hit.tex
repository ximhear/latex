\documentclass[a4paper, 12pt]{article}
\usepackage{amsmath}
\usepackage{amssymb}
\usepackage{kotex}  % 한국어 지원을 위한 패키지
\usepackage{graphicx}
\usepackage{hyperref}

\title{레이와 구체의 충돌 여부를 판단하는 수식 유도}
\author{}
\date{}

\begin{document}

\maketitle

\section{레이와 구체의 충돌 여부 판단}

레이와 구체의 충돌 여부를 판단하는 판별식(discriminant)을 유도하기 위해서는 먼저 레이와 구체의 수학적 표현을 이해하고, 이를 결합하여 충돌 조건을 도출해야 합니다.

\subsection{레이의 수학적 표현}

레이(Ray)는 3D 공간에서 하나의 시작점(원점)과 방향으로 정의됩니다. 레이는 다음과 같이 표현할 수 있습니다:

\[
\mathbf{P}(t) = \mathbf{O} + t \cdot \mathbf{D}
\]

여기서,
\begin{itemize}
    \item \(\mathbf{P}(t)\)는 레이 상의 한 점을 나타냅니다.
    \item \(\mathbf{O}\)는 레이의 시작점(원점, origin)입니다.
    \item \(\mathbf{D}\)는 레이의 방향 벡터(direction)입니다.
    \item \(t\)는 매개변수로, 레이의 시작점에서부터 얼마나 멀리 떨어져 있는지를 나타냅니다. \(t\)는 음수일 수도 있고, 양수일 수도 있습니다.
\end{itemize}

\subsection{구체의 수학적 표현}

구체(Sphere)는 중심점 \(\mathbf{C}\)과 반지름 \(r\)으로 정의됩니다. 구체는 다음 방정식으로 나타낼 수 있습니다:

\[
(\mathbf{P} - \mathbf{C}) \cdot (\mathbf{P} - \mathbf{C}) = r^2
\]

여기서,
\begin{itemize}
    \item \(\mathbf{P}\)는 구체 표면 위의 한 점입니다.
    \item \(\mathbf{C}\)는 구체의 중심점(center)입니다.
    \item \(r\)는 구체의 반지름(radius)입니다.
    \item \(\cdot\)은 벡터의 내적(dot product)을 의미합니다.
\end{itemize}

\subsection{레이-구체 충돌 문제}

레이와 구체가 만난다는 것은 레이의 어떤 점 \(\mathbf{P}(t)\)가 구체의 표면 위에 있다는 것을 의미합니다. 따라서, \(\mathbf{P}(t)\)를 구체의 방정식에 대입하여 충돌 여부를 판별할 수 있습니다.

레이의 표현 \(\mathbf{P}(t) = \mathbf{O} + t \cdot \mathbf{D}\)를 구체의 방정식에 대입하면:

\[
(\mathbf{O} + t \cdot \mathbf{D} - \mathbf{C}) \cdot (\mathbf{O} + t \cdot \mathbf{D} - \mathbf{C}) = r^2
\]

이를 전개하면:

\[
(\mathbf{O} - \mathbf{C}) \cdot (\mathbf{O} - \mathbf{C}) + 2t \cdot (\mathbf{D} \cdot (\mathbf{O} - \mathbf{C})) + t^2 \cdot (\mathbf{D} \cdot \mathbf{D}) = r^2
\]

이 방정식은 \(t\)에 대한 2차 방정식입니다. 표준적인 2차 방정식의 형태로 정리하면:

\[
t^2 \cdot (\mathbf{D} \cdot \mathbf{D}) + 2t \cdot (\mathbf{D} \cdot (\mathbf{O} - \mathbf{C})) + \left((\mathbf{O} - \mathbf{C}) \cdot (\mathbf{O} - \mathbf{C}) - r^2\right) = 0
\]

이를 \(t\)에 대한 일반적인 2차 방정식의 형태로 나타내면:

\[
a t^2 + b t + c = 0
\]

여기서,

\begin{itemize}
    \item \(a = \mathbf{D} \cdot \mathbf{D}\): 레이 방향 벡터 \(\mathbf{D}\)의 자기 내적, 즉 방향 벡터의 크기의 제곱입니다. 일반적으로 방향 벡터는 정규화되어 있으므로 \(a = 1\)이 됩니다.
    \item \(b = 2 \cdot (\mathbf{D} \cdot (\mathbf{O} - \mathbf{C}))\): 레이의 방향 벡터와 구체 중심에서 레이의 시작점으로 향하는 벡터 간의 내적에 2를 곱한 값입니다.
    \item \(c = (\mathbf{O} - \mathbf{C}) \cdot (\mathbf{O} - \mathbf{C}) - r^2\): 레이 시작점에서 구체 중심까지의 거리의 제곱에서 구체의 반지름의 제곱을 뺀 값입니다.
\end{itemize}

\subsection{판별식(Discriminant)의 유도}

2차 방정식의 해는 다음과 같은 판별식으로 구할 수 있습니다:

\[
t = \frac{-b \pm \sqrt{b^2 - 4ac}}{2a}
\]

여기서 중요한 부분은 루트 안의 표현인 \(b^2 - 4ac\)입니다. 이것을 \textbf{판별식(discriminant)}이라고 합니다.

\begin{itemize}
    \item \textbf{판별식 \(> 0\)}: 레이와 구체가 두 점에서 만난다는 의미입니다. 레이가 구체를 통과합니다.
    \item \textbf{판별식 \(= 0\)}: 레이와 구체가 한 점에서 만난다는 의미입니다. 레이가 구체에 접합니다.
    \item \textbf{판별식 \(< 0\)}: 레이와 구체가 만나지 않는다는 의미입니다. 레이는 구체를 비켜갑니다.
\end{itemize}

따라서, 함수에서는 판별식 \(b^2 - 4ac\)의 값이 0보다 큰지를 검사하여 레이와 구체의 충돌 여부를 판단합니다:

\begin{verbatim}
float discriminant = b * b - 4.0 * a * c;
return discriminant > 0;
\end{verbatim}

\section{요약}

레이-구체 충돌 문제는 레이와 구체의 수학적 방정식을 결합하여 2차 방정식의 형태로 나타냅니다. 이 2차 방정식의 해의 존재 여부를 판별식(discriminant)을 통해 판단하며, 이 값이 양수이면 레이와 구체가 교차하고, 음수이면 교차하지 않는다는 결론을 내립니다.

\end{document}
