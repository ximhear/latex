\documentclass{article}
\usepackage{kotex}
\usepackage{amsmath}

\begin{document}
   
\section{@Observable}
Observable의 property가 변경될 경우, View에서 the property를 참조한다면 View \textbf{전체}(All raw child views in the view)가 refresh된다.

$\rightarrow$ Child View에도 동일하게 적용된다. 즉, the property를 참조한다면 Child View 전체가 refresh된다.

$\rightarrow$ Parent View에도 동일하게 적용된다. $\rightarrow$ Child View에도 동일하게 적용된다.

\section{ObservableObject}
\begin{enumerate}
\item ObservableObject의 @Published property가 변경될 경우, the ObservableObject를 StateObject나 ObservedObject로 가진 View의 \textbf{전체}가 refresh된다.
\item 변경된 property를 참조하지 않아도 StateObject나 ObservedObject로 가지고 있으면 refresh된다.
\item ObservableObject를 StateObject나 ObservedObject가 아닌 일반 var, let으로 가지고 있다면 refresh되지 않는다.
\end{enumerate}

\section{@State}
@State or @Binding 변수가 바뀌면 해당 View의 전체(raw view만)가 refresh된다.

$\rightarrow$ Child View가 Binding을 가진 경우, the Child View의 전체가 refresh된다. 

$\rightarrow$ Parent View가 State or Binding을 가진 경우, the Parent View의 전체가 refresh된다. 

\section{용어설명}
\texttt{raw view} : 개발자가 View를 상속해서 만들지 않은 View, 일반 Text, Button 등.
\texttt{전체} : View에 포함된 모든 raw view들. 

\end{document}   
