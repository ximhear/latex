\documentclass{article}
\usepackage{kotex}
\usepackage{amsmath}
\usepackage{listings}
\usepackage{color}
\usepackage{algorithm}
\usepackage{algorithmic}

% 코드 하이라이팅을 위한 설정
\definecolor{codegreen}{rgb}{0,0.6,0}
\definecolor{codegray}{rgb}{0.5,0.5,0.5}
\definecolor{codepurple}{rgb}{0.58,0,0.82}
\definecolor{backcolour}{rgb}{0.95,0.95,0.92}

\lstdefinestyle{mystyle}{
    backgroundcolor=\color{backcolour},   
    commentstyle=\color{codegreen},
    keywordstyle=\color{magenta},
    numberstyle=\tiny\color{codegray},
    stringstyle=\color{codepurple},
    basicstyle=\footnotesize,
    breakatwhitespace=false,         
    breaklines=true,                 
    captionpos=b,                    
    keepspaces=true,                 
    numbers=left,                    
    numbersep=5pt,                  
    showspaces=false,                
    showstringspaces=false,
    showtabs=false,                  
    tabsize=2
}

\lstset{style=mystyle}

\begin{document}
   
\section{@Observable}
Observable의 property가 변경될 경우, View에서 the property를 참조한다면 View \textbf{전체}(All raw child views in the view)가 refresh된다.

$\rightarrow$ Child View에도 동일하게 적용된다. 즉, the property를 참조한다면 Child View 전체가 refresh된다.

$\rightarrow$ Parent View에도 동일하게 적용된다. $\rightarrow$ Child View에도 동일하게 적용된다.

\subsection{Obervable in Observable}
Observable내에 Observable property가 있는 경우를 보자.
아래 코드의 StateRefreshTestModel.data1.count가 변경될 경우, 이를 참조하는 View의 전체가 refresh된다.

data1이 @ObservationIgnored로 선언되어 있어도 count가 변경되면 refresh된다.

이것으로 보아 다음과 같은 결론을 내릴 수 있다.

\textbf{Perperty 체인의 최종 property가 ObservationIgnored가 아니면 중간에 ObservationIgnored의 유무에 상관없이 refresh된다.}

\begin{lstlisting}[language=Swift]
    @Observable class StateRefreshTestModel {
        var count: Int = 0
        @ObservationIgnored var count1: Int = 0
        func increase() {
            count += 1
        }
        
        func increase1() {
            count1 += 1
        }
        
        @ObservationIgnored var data1 = Data1()
    }
    
    @Observable class Data1 {
        var count: Int = 0
        var count1: Int = 0
        @ObservationIgnored var count2: Int = 0
        func increase() {
            count += 1
        }
        func increase1() {
            count1 += 1
        }
        func increase2() {
            count2 += 1
        }
    }
\end{lstlisting}


\section{ObservableObject}
\begin{enumerate}
\item ObservableObject의 @Published property가 변경될 경우, the ObservableObject를 StateObject나 ObservedObject로 가진 View의 \textbf{전체}가 refresh된다.
\item 변경된 property를 참조하지 않아도 StateObject나 ObservedObject로 가지고 있으면 refresh된다.
\item ObservableObject를 StateObject나 ObservedObject가 아닌 일반 var, let으로 가지고 있다면 refresh되지 않는다.
\end{enumerate}

\section{@State}
@State or @Binding 변수가 바뀌면 해당 View의 전체(raw view만)가 refresh된다.

$\rightarrow$ Child View가 Binding을 가진 경우, the Child View의 전체가 refresh된다. 

$\rightarrow$ Parent View가 State or Binding을 가진 경우, the Parent View의 전체가 refresh된다. 

\section{용어설명}
\texttt{raw view} : 개발자가 View를 상속해서 만들지 않은 View, 일반 Text, Button 등.
\texttt{전체} : View에 포함된 모든 raw view들. 

\end{document}   
